Three different experiments were run for this assignment. 

\paragraph{Base experiment} This configuration had a pill density of 50\%, a fruit spawn probability of 1\%, a fruit score of 10, and a time multiplier of 2 for the Pac-Man game.
Additionally, for the GP configuration parameters $\mu_{Pac-Man},\mu_{Ghost}$ were both set to 200 and $\lambda_{Pac-Man}, \lambda_{Ghost}$ were set to 100. Both the Ghost population and the Pac-Man population used truncation for survival selection and fitness proportional selection for parent selection. $p_{Pac-Man},p_{Ghost}$ were both set to 100\% and the number of evals till termination is set to 2000.

\paragraph{High Pill Density}
For this configuration, the only change was to the pill density which was 80\%.
As seen in 2B a higher pill density usually correlates to higher fitness.
This does not entirely mean a smarter Pac-Man as there are more pills for him to accidentally step on the higher the density is.
Regardless, the author thought it would be interesting to look at.

\paragraph{Max Pill Density}

Using the same logic here we set the pill density to 100\% to see if it further improves anything.