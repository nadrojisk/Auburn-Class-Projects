Utilizing competitive co-evaluation one can create smart and more diverse solutions to problems.
Before when it was a GP controller against a random agent, not a lot of interesting things occurred.
However, when both agents can learn and play off of each other interesting strategies can occur.
For instance, in one game the author saw a Ghost evolve with a tree that simply added together with the shortest path to Pac-Man to the shortest path to Pac-Man.
Utilizing this the Ghost ensured that they would always move closer to Pac-Man acting like a homing missing and eating him.
However, this strategy only seemed to work against `dumb' Pac-Man, as the author saw one game where the Ghost utilized this strategy, however, Pac-Man simply kept moving away from the Ghost and kept consuming pills so he was able to finish the game with the Ghost one tile away from him.
The issue with that Ghost strategy is that \emph{all} of the Ghosts take the same move, however, if they had split off they could have potentially boxed Pac-Man in.
Regardless it was very interesting to watch.

There were some areas the author wished to look into but ran out of time and motivation (mainly due to the end of the semester burnout).
An example of an area that the author wished to implement was to have paired the controllers up multiple times instead of doing a 1-1 evaluation as doing 1-1 under samples both populations.
For instance, a Pac-Man controller that is just adequate could get paired against a really bad Ghost controller. 
In this pairing, the Pac-Man controller could achieve a high score simply because the Ghost controller is not intelligent enough to do anything.
If this happens then the Pac-Man controller would achieve high fitness even though it is not that great.
Another area that needs improvement is the optimization of the codebase.
Run-time was a reoccurring issue throughout the entire project and the author understands that it is usually an issue with AI-like projects.
However, running a debugger and waiting for the program to execute for 30 minutes just for it to break was unfortunate.
This is not of course the fault of the TA's or the professor but the student himself. 

The projects were all very interesting throughout the semester and the author very much enjoyed working on them.
Towards the end, some slacking did occur and they may have not been in the best state.
The TA's also did a great job of answering any pending questions on the projects and for just general EC stuff!
